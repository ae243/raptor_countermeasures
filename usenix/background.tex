\section{Background and Related Work}
Here we discuss network level adversaries on the Tor network and past work on defending against such network level adversaries. 

\subsection{Network Adversaries on Tor}
%To communicate with a destination, Tor clients establish
%layered circuits through three subsequent Tor relays. The
%three relays are referred to as: entry (or guard) for the
%first one, middle for the second one, and exit relay for
%the last one. To load balance its traffic, Tor clients select
%relays with a probability that is proportional to their
%network capacity. Encryption is used to ensure that each
%relay learns the identity of only the previous hop and the
%next hop in the communications, and no single relay can
%link the client to the destination server.
%
%It is well known that if an attacker can observe the
%traffic from the destination server to the exit relay as well
%as from the entry relay to the client (or traffic from the
%client to the entry relay and from the exit relay to the
%destination server), then it can leverage correlation between
%packet timing and sizes to infer the network identities
%of clients and servers (end-to-end timing analysis).
%This timing analysis works even if the communication is
%encrypted~\cite{sun2015raptor}.
%Network-level adversaries are known in anonymity networks. 
Feamster and Dingledine \cite{feamster2004location} first investigated AS-level adversaries in anonymity networks, and they showed that some ASes could appear on nearly 30\% of entry-exit pairs. Murdoch and Zielinski \cite{murdoch2007sampled} later demonstrated the threat posed by network-level adversaries who can deanonymize users by performing traffic analysis. Furthermore, Edman and Syverson \cite{edman2009awareness} demonstrated that even given the explosive growth of Tor during the past years, still about 18\% of Tor circuits result in a single AS being able to observe both ends of the communication path. In 2013, Johnson \emph{et al.} \cite{johnson2013users} evaluated the security of Tor users over a period of time, and the results indicated that a network-level adversary with just a low-bandwidth cost could deanonymize any user within three months with over 50\% probability and within six months with over 80\% probability.

While all prior research, to our knowledge, focus on passive adversaries, more recently, Sun \emph{et al.} ~\cite{sun2015raptor} discovered the threat posed by active AS-level adversaries, who can perform active BGP routing attacks to put themselves onto the path between client-entry and/or exit-destination, and thus be able to observe Tor traffic. The paper also showed from past BGP data that during past known real-world BGP prefix attacks, Tor relays were affected as well - i.e., in the Indosat hijack in 2014, among the victim prefixes there were 44 Tor relays, and 33 of them were guard relays which had direct connections with Tor clients. 

\subsection{Tor Path Selection against Network Adversaries}
The existence of network-level adversaries motivates the research on AS-awareness in path selection in Tor. In 2012, Akhoondi \emph{et al.} \cite{akhoondi2012lastor} proposed LASTor, a Tor client which takes into account AS-level path and relay locations in selecting a path; although our work differs by considering relays' resilience to active attacks and relays' capacity. Recently, Nithyanand \emph{et al.} \cite{starov2015measuring} constructed a new Tor client, Astoria, which adopted a new path selection algorithm that considered more aspects - relay capacity, asymmetric routing, and colluding ASes. However, Astoria only considers a passive AS-level attacker and does not consider the case of active routing attacks.\\
\\

This motivates our work on defending Tor against active AS-level adversaries who can launch active routing attacks such as BGP prefix hijacks and interceptions. Towards this goal, it is important to understand AS-level Internet topology and network path predictions. Lad \emph{et al.} \cite{lad2007understanding} investigated the relationship between Internet topology and prefix hijacking, and provided a metric for evaluating AS resilience to active prefix hijack attacks. Although the study was conducted in 2007 when there were far less ASes than now, and the study only simulated a partial attack scenario of 1000 randomly selected ASes as attackers, it provides a foundational starting point for our work. Thus, in section \ref{hijack_interception_measurement}, we will start with measuring the AS resilience of the Tor network to active prefix hijack attacks using the metric, but considering \emph{all} attack scenarios; then, we will further extend the metric to evaluate AS resilience to active prefix \emph{interception} attacks, a novel metric and measurement. In Section \ref{subsec:relayselection}, we incorporate the AS resilience metric into the Tor guard relay selection algorithm. 


%RAPTOR attacks are a suite of attacks that can be launched by 
%Autonomous Systems (ASes) to compromise user anonymity.  There are 
%three different types of attacks in this classification.
%
%{\bf Asymmetric Traffic Analysis.} AS-level adversaries can
%exploit the asymmetric nature of Internet routing to increase
%the chance of observing at least one direction of
%user traffic at both ends of the communication.
%
%{\bf Natural Churn.} AS-level adversaries can exploit natural churn in Internet
%routing to lie on the BGP paths for more users over
%time.
%
%{\bf BGP Hijacks.} Strategic AS-level adversaries can manipulate Internet
%routing via BGP hijacks (to discover the users using
%specific Tor guard nodes) and interceptions (to perform
%traffic analysis).
