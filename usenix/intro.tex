\section{Introduction}

The Tor network ~\cite{dingledine2004tor} has been the most widely used system for anonymous communications that protect user identity from untrusted parties who have access to user traffic. Tor serves millions of users and carries terabytes of traffic everyday with its network of over 7,000 relays ~\cite{tormetrics}, which makes it a popular target for adversaries who wish to break the anonymity of the users. 

Tor is known to be vulnerable to traffic correlation attacks. An adversary who can observe the traffic at both ends of the communication (i.e., between Tor client and the entry relay, and between exit relay and the destination server) can perform  correlation analysis on packet sizes and timings to deanonymize the Tor users ~\cite{shmatikov2006timing} \cite{syverson2001towards}. Network-level adversaries, i.e., autonomous systems (ASes), that lie on the path between Tor client to the entry relay and exit relay to the destination server have been shown to be at a compromising position to deanonymize Tor clients ~\cite{feamster2004location}, \cite{edman2009awareness}, \cite{johnson2013users}. More recently, researchers have further exploited the dynamics of BGP routing that exaggerated this threat by enabling more network-level adversaries to be at the compromising position \cite{sun2015raptor}, including active BGP prefix attacks which were not being studied previously on Tor. 

Building countermeasures to defend Tor against such malicious AS-level adversaries is an urgent challenge facing the research community. Past works have explored AS-aware relay selection algorithms that minimize the chance of selecting Tor relays with the same AS lying on both ends of the communication paths ~\cite{edman2009awareness}, \cite{akhoondi2012lastor}, \cite{starov2015measuring}. However, all these work only focus on mitigating \emph{passive} attacks in which AS-level adversaries only passively observe traffic instead of launching any \emph{active} attacks. It has been shown that active BGP routing attacks can pose new threats on Tor users and there were Tor relays that already got affected in past real-world BGP attacks \cite{sun2015raptor}. Thus, these motivated our work on developing countermeasures against such active BGP attacks on Tor which have been previously understudied. 

In this paper, our contributions consist of three parts. First, we quantify the vulnerability of the current Tor network to active BGP prefix hijacks and interceptions. Second, we develop proactive approaches to lower the probability of being affected by such attacks, which includes a novel Tor guard relay selection algorithm. Finally, we present a live monitoring framework on Tor that can detect routing anomalies in Tor relays. 
\\
\textbf{Measurement on the Tor network.} We measure the vulnerability of the current Tor network by calculating the resilience to BGP prefix attacks for all ASes that contain Tor guard/exit relays. Based on the current internet topology ~\cite{topology} and Tor consensus data ~\cite{torconsensus}, we first leverage the AS resilience metric ~\cite{lad2007understanding} to measure resilience to \emph{all} possible hijacking scenarios. Then, we further extend the metric with more complicated scenarios to measure resilience to interception attacks launched by Tier 1 ASes. Finally, we perform a case study on the state of Tor resilience during a past known attack - the Indosat hijack in 2014. Our keys findings are:
\begin{itemize}
\item Resilience values corresponding to hijack attacks for all Tor-related ASes are similar to a normal distribution, where most ASes fall in the middle of the spectrum. However, some ASes which are responsible for high number of relays and/or high bandwidths have low resilience values, i.e., AS 16276 (OVH) which contains 339 Tor relays only has resilience value of 0.32 on a scale of $[0,1]$.
\item Resilience values corresponding to interception attacks for all Tor-related ASes are skewed towards higher resilience. However, similar to that in hijack resilience, some ASes (i.e., OVH) carrying high bandwidths have relatively low resilience. 
\item Average Tor AS resilience has increased each year from 2008 to 2016. 
\item 73\% of the Tor-related ASes that were affected in Indosat hijack in 2014 had resiliency values $< 0.6$.
\end{itemize}
\textbf{Proactive approaches.} First, we start a campaign by contacting Tor relay operators to move their relays into a more specific prefix range, i.e., /24. We have successfully cooperated with Princeton University to move a Tor relay into a /24 prefix from the original /16 prefix. Second, we propose and implement a novel Tor guard relay selection algorithm which incorporates AS resiliency of the relays into considerations. Our guard relay selection algorithm is first algorithm to consider resilience to active BGP routing attacks on Tor ~\cite{sun2015raptor}. The algorithm combines resiliency and bandwidth into relay selection to ensure security as well as performance. Our evaluation shows that the algorithm achieves 35\% reduction in probability of being affected by a prefix hijack attack and 49\% improvement on sender anonymity bound compared to the current Tor relay selection algorithm. At the same time, it does not suffer any performance loss based on a real world evaluation on page load time from Alexa Top 100 sites. 
\\
\textbf{Reactive approaches.} We build a live monitoring framework that monitors routing activities on Tor relays in real time. The monitoring framework consists of two parts: BGP monitoring framework which monitors the control plane, and Traceroute monitoring framework which monitors the data plane. BGP monitoring framework collects live BGP announcement data from BGP Stream ~\cite{bgpstream}, in combination with latest hourly Tor relay data, and detects any routing anomaly in real time. Traceroute monitoring framework utilizes Planetlab nodes to actively send traceroute requests to a selective set of Tor relays (i.e., high bandwidth relays) to monitor routing anomaly in the data plane. We believe our live monitoring framework will help enhance the transparency of the Tor network against active BGP attacks.
\\
The paper is organized as follows. Section 2 provides a brief overview of background and related work on Tor.  Section 3 describes the metrics and methodology used to measure Tor relay resilience to active BGP prefix hijack and interception attacks. Section 4 discusses our campaign on moving Tor relays to /24 prefixes and presents our new Tor guard relay selection algorithm. Section 5 demonstrates our design for the live monitoring framework and describes deployment experiences. Section 6 discusses potential obstacles and shortcomings of the current approaches and directions for future work. Finally, we conclude in Section 7. 
%Recently, researchers have shown that AS-level adversaries can exploit the dynamics of BGP routing and launch new attacks on Tor \cite{sun2015raptor}, including both passive attacks exploiting routing asymmetry and natural churn, as well as active attacks with BGP prefix hijacking/interception. Thus, these new attacks urge the need to build countermeasures to defend Tor against such malicious AS-level adversaries. In this work, we will focus on developing countermeasures against active BGP routing attacks. 
%
%There are two potential ways to counter the attacks: (1) Mitigating traffic interception, and (2) Mitigating correlation attacks. However, mitigating correlation attacks usually involve employing extra encryption schemes and/or packet obfuscation, which either requires extensive engineering efforts, or could result in high latency of Tor. Thus, given the constraints of mitigating correlation attacks, we will build countermeasures that focus on mitigating traffic interception, more specifically, against active BGP routing attacks. Our approach includes two parts, as following. 
%
%The first part is a set of proactive approaches to countering BGP attacks.  These approaches make it more difficult for an attacker to hijack Tor traffic.  There's a surprisingly large number of Tor relays that are announced in prefixes that are larger than /24, which means they are vulnerable to sub-prefix hijack attacks.  Therefore, one countermeasure is for operators to announce Tor relays in a /24.  Similarly, operators can use a static route to guard relays, so that the traffic cannot be hijacked.  In the case of equal-length prefix hijack attacks, we also propose a new path selection algorithm for the path from the client to the guard relay.  
%
%The second part is a reactive approach: a monitoring framework that can detect attacks in real-time.  We have built a BGP monitoring framework for the Tor network that uses multiple different techniques to detect suspicious prefix announcements.  This is used in conjunction with a traceroute monitoring framework for validation of path changes and suspicious paths.  
%
%Previous work has shown that the Tor network is susceptible to AS-level adversaries, and specifically prefix hijacks.  In this work, we measure how resilient each AS -- with at least one Tor relay -- is to a prefix hijack from anywhere else on the Internet.  Our contributions are:

%\begin{itemize}
%\item Measure hijack resiliency of the ASes that contain relays and compare the resiliency of ASes that contain more relays to those that contain few relays.
%\item Measure interception resiliency of the ASes that contain relays and compare the resiliency of ASes that contain more relays to those that contain few relays.
%\item Quantify how much resiliency to hijacks and interceptions on the Tor network differs year to year, starting in 2008.
%%\item Quantify if Tor relays have become more resilient since the initial network was built.
%%\item Quantify how fast relay resilience changes.
%\item Build a real-time monitoring system for the Tor network, which uses both the control plane and data plane.
%\item Develop a new guard selection technique for Tor clients, which we evaluate and show is more resistent to hijack attacks.
%\item Discuss experiences with relay operators in regards to announcing relays in a /24.
%\end{itemize}


