% TEMPLATE for Usenix papers, specifically to meet requirements of
%  USENIX '05
% originally a template for producing IEEE-format articles using LaTeX.
%   written by Matthew Ward, CS Department, Worcester Polytechnic Institute.
% adapted by David Beazley for his excellent SWIG paper in Proceedings,
%   Tcl 96
% turned into a smartass generic template by De Clarke, with thanks to
%   both the above pioneers
% use at your own risk.  Complaints to /dev/null.
% make it two column with no page numbering, default is 10 point

% Munged by Fred Douglis <douglis@research.att.com> 10/97 to separate
% the .sty file from the LaTeX source template, so that people can
% more easily include the .sty file into an existing document.  Also
% changed to more closely follow the style guidelines as represented
% by the Word sample file. 

% Note that since 2010, USENIX does not require endnotes. If you want
% foot of page notes, don't include the endnotes package in the 
% usepackage command, below.

% This version uses the latex2e styles, not the very ancient 2.09 stuff.
\documentclass[letterpaper,twocolumn,10pt]{article}
\usepackage{usenix,epsfig,endnotes}
\usepackage{graphicx}
\usepackage{url}
\usepackage{listings}
\usepackage{subfigure}
\usepackage{paralist}
\usepackage{textcomp}
\usepackage{xspace}
\usepackage{ifthen}
\usepackage{amsmath}
\usepackage{amssymb}
\usepackage{color}
\usepackage{algpseudocode}
\usepackage{algorithm}

\newcommand{\yixin}[1]{\emph{\color{red}(#1)}}
\newcommand{\annie}[1]{\emph{\color{blue}(#1)}}


\newcommand{\comment}[1]{}

\begin{document}

%don't want date printed
\date{}

%make title bold and 14 pt font (Latex default is non-bold, 16 pt)
\title{\Large \bf RAPTOR Countermeasures}

%for single author (just remove % characters)
%\author{
%{}\\
%Princeton University
%\and
%{}\\
%Princeton University
%\and
%{}\\
%Princeton University
% copy the following lines to add more authors
% \and
% {\rm Name}\\
%Name Institution
%} % end author

\maketitle

% Use the following at camera-ready time to suppress page numbers.
% Comment it out when you first submit the paper for review.
\thispagestyle{empty}

\subsection*{Abstract}
Tor is a widely-used anonymity system, but has also been shown to be vulnerable to 
different types of traffic analysis attacks.  Recently, experiments have shown that 
Tor is susceptible to asymmetric traffic analysis attacks, as well as other attacks 
at the AS level.  These include malicious manipulations of traffic such as BGP prefix 
hijack and interception attacks.  
This paper presents a new evaluation of how resilient the Tor network is to prefix 
hijack attacks.  We also present a set of novel countermeasures that will help 
prevent these types of attacks from being successful: a new path selection algorithm, 
a live monitoring framework that uses a variety of heuristics for attack detection, and 
suggestions for Tor relay operators. 

\section{Introduction}

The Tor network ~\cite{dingledine2004tor} has been the most widely used system for anonymous communications that protect user identity from untrusted parties who have access to user traffic. Tor serves millions of users and carries terabytes of traffic everyday with its network of over 7,000 relays ~\cite{tormetrics}, which makes it a popular target for adversaries who wish to break the anonymity of the users. 

Tor is known to be vulnerable to traffic correlation attacks. An adversary who can observe the traffic at both ends of the communication (i.e., between Tor client and the entry relay, and between exit relay and the destination server) can perform  correlation analysis on packet sizes and timings to deanonymize the Tor users ~\cite{shmatikov2006timing} \cite{syverson2001towards}. Network-level adversaries, i.e., autonomous systems (ASes), that lie on the path between Tor client to the entry relay and exit relay to the destination server have been shown to be at a compromising position to deanonymize Tor clients ~\cite{feamster2004location}, \cite{edman2009awareness}, \cite{johnson2013users}. More recently, researchers have further exploited the dynamics of BGP routing that exaggerated this threat by enabling more network-level adversaries to be at the compromising position \cite{sun2015raptor}, including active BGP prefix attacks which were not being studied previously on Tor. 

Building countermeasures to defend Tor against such malicious AS-level adversaries is an urgent challenge facing the research community. Past works have explored AS-aware relay selection algorithms that minimize the chance of selecting Tor relays with the same AS lying on both ends of the communication paths ~\cite{edman2009awareness}, \cite{akhoondi2012lastor}, \cite{starov2015measuring}. However, all these work only focus on mitigating \emph{passive} attacks in which AS-level adversaries only passively observe traffic instead of launching any \emph{active} attacks. It has been shown that active BGP routing attacks can pose new threats on Tor users and there were Tor relays that already got affected in past real-world BGP attacks \cite{sun2015raptor}. Thus, these motivated our work on developing countermeasures against such active BGP attacks on Tor which have been previously understudied. 

In this paper, our contributions consist of three parts. First, we quantify the vulnerability of the current Tor network to active BGP prefix hijacks and interceptions. Second, we develop proactive approaches to lower the probability of being affected by such attacks, which includes a novel Tor guard relay selection algorithm. Finally, we present a live monitoring framework on Tor that can detect routing anomalies in Tor relays. 
\\
\textbf{Measurement on the Tor network.} We measure the vulnerability of the current Tor network by calculating the resilience to BGP prefix attacks for all ASes that contain Tor guard/exit relays. Based on the current internet topology ~\cite{topology} and Tor consensus data ~\cite{torconsensus}, we first leverage the AS resilience metric ~\cite{lad2007understanding} to measure resilience to \emph{all} possible hijacking scenarios. Then, we further extend the metric with more complicated scenarios to measure resilience to interception attacks launched by Tier 1 ASes. Finally, we perform a case study on the state of Tor resilience during a past known attack - the Indosat hijack in 2014. Our keys findings are:
\begin{itemize}
\item Resilience values corresponding to hijack attacks for all Tor-related ASes are similar to a normal distribution, where most ASes fall in the middle of the spectrum. However, some ASes which are responsible for high number of relays and/or high bandwidths have low resilience values, i.e., AS 16276 (OVH) which contains 339 Tor relays only has resilience value of 0.32 on a scale of $[0,1]$.
\item Resilience values corresponding to interception attacks for all Tor-related ASes are skewed towards higher resilience. However, similar to that in hijack resilience, some ASes (i.e., OVH) carrying high bandwidths have relatively low resilience. 
\item Average Tor AS resilience has increased each year from 2008 to 2016. 
\item 73\% of the Tor-related ASes that were affected in Indosat hijack in 2014 had resiliency values $< 0.6$.
\end{itemize}
\textbf{Proactive approaches.} First, we start a campaign by contacting Tor relay operators to move their relays into a more specific prefix range, i.e., /24. We have successfully cooperated with Princeton University to move a Tor relay into a /24 prefix from the original /16 prefix. Second, we propose and implement a novel Tor guard relay selection algorithm which incorporates AS resiliency of the relays into considerations. Our guard relay selection algorithm is first algorithm to consider resilience to active BGP routing attacks on Tor ~\cite{sun2015raptor}. The algorithm combines resiliency and bandwidth into relay selection to ensure security as well as performance. Our evaluation shows that the algorithm achieves 35\% reduction in probability of being affected by a prefix hijack attack and 49\% improvement on sender anonymity bound compared to the current Tor relay selection algorithm. At the same time, it does not suffer any performance loss based on a real world evaluation on page load time from Alexa Top 100 sites. 
\\
\textbf{Reactive approaches.} We build a live monitoring framework that monitors routing activities on Tor relays in real time. The monitoring framework consists of two parts: BGP monitoring framework which monitors the control plane, and Traceroute monitoring framework which monitors the data plane. BGP monitoring framework collects live BGP announcement data from BGP Stream ~\cite{bgpstream}, in combination with latest hourly Tor relay data, and detects any routing anomaly in real time. Traceroute monitoring framework utilizes Planetlab nodes to actively send traceroute requests to a selective set of Tor relays (i.e., high bandwidth relays) to monitor routing anomaly in the data plane. We believe our live monitoring framework will help enhance the transparency of the Tor network against active BGP attacks.
\\
The paper is organized as follows. Section 2 provides a brief overview of background and related work on Tor.  Section 3 describes the metrics and methodology used to measure Tor relay resilience to active BGP prefix hijack and interception attacks. Section 4 discusses our campaign on moving Tor relays to /24 prefixes and presents our new Tor guard relay selection algorithm. Section 5 demonstrates our design for the live monitoring framework and describes deployment experiences. Section 6 discusses potential obstacles and shortcomings of the current approaches and directions for future work. Finally, we conclude in Section 7. 
%Recently, researchers have shown that AS-level adversaries can exploit the dynamics of BGP routing and launch new attacks on Tor \cite{sun2015raptor}, including both passive attacks exploiting routing asymmetry and natural churn, as well as active attacks with BGP prefix hijacking/interception. Thus, these new attacks urge the need to build countermeasures to defend Tor against such malicious AS-level adversaries. In this work, we will focus on developing countermeasures against active BGP routing attacks. 
%
%There are two potential ways to counter the attacks: (1) Mitigating traffic interception, and (2) Mitigating correlation attacks. However, mitigating correlation attacks usually involve employing extra encryption schemes and/or packet obfuscation, which either requires extensive engineering efforts, or could result in high latency of Tor. Thus, given the constraints of mitigating correlation attacks, we will build countermeasures that focus on mitigating traffic interception, more specifically, against active BGP routing attacks. Our approach includes two parts, as following. 
%
%The first part is a set of proactive approaches to countering BGP attacks.  These approaches make it more difficult for an attacker to hijack Tor traffic.  There's a surprisingly large number of Tor relays that are announced in prefixes that are larger than /24, which means they are vulnerable to sub-prefix hijack attacks.  Therefore, one countermeasure is for operators to announce Tor relays in a /24.  Similarly, operators can use a static route to guard relays, so that the traffic cannot be hijacked.  In the case of equal-length prefix hijack attacks, we also propose a new path selection algorithm for the path from the client to the guard relay.  
%
%The second part is a reactive approach: a monitoring framework that can detect attacks in real-time.  We have built a BGP monitoring framework for the Tor network that uses multiple different techniques to detect suspicious prefix announcements.  This is used in conjunction with a traceroute monitoring framework for validation of path changes and suspicious paths.  
%
%Previous work has shown that the Tor network is susceptible to AS-level adversaries, and specifically prefix hijacks.  In this work, we measure how resilient each AS -- with at least one Tor relay -- is to a prefix hijack from anywhere else on the Internet.  Our contributions are:

%\begin{itemize}
%\item Measure hijack resiliency of the ASes that contain relays and compare the resiliency of ASes that contain more relays to those that contain few relays.
%\item Measure interception resiliency of the ASes that contain relays and compare the resiliency of ASes that contain more relays to those that contain few relays.
%\item Quantify how much resiliency to hijacks and interceptions on the Tor network differs year to year, starting in 2008.
%%\item Quantify if Tor relays have become more resilient since the initial network was built.
%%\item Quantify how fast relay resilience changes.
%\item Build a real-time monitoring system for the Tor network, which uses both the control plane and data plane.
%\item Develop a new guard selection technique for Tor clients, which we evaluate and show is more resistent to hijack attacks.
%\item Discuss experiences with relay operators in regards to announcing relays in a /24.
%\end{itemize}



\section{Background}
Here we discuss some background on the Tor network as well as RAPTOR attacks~\cite{sun2015raptor}.

\subsection{Tor}
To communicate with a destination, Tor clients establish
layered circuits through three subsequent Tor relays. The
three relays are referred to as: entry (or guard) for the
first one, middle for the second one, and exit relay for
the last one. To load balance its traffic, Tor clients select
relays with a probability that is proportional to their
network capacity. Encryption is used to ensure that each
relay learns the identity of only the previous hop and the
next hop in the communications, and no single relay can
link the client to the destination server.

It is well known that if an attacker can observe the
traffic from the destination server to the exit relay as well
as from the entry relay to the client (or traffic from the
client to the entry relay and from the exit relay to the
destination server), then it can leverage correlation between
packet timing and sizes to infer the network identities
of clients and servers (end-to-end timing analysis).
This timing analysis works even if the communication is
encrypted.

\subsection{RAPTOR Attacks}
RAPTOR attacks are a suite of attacks that can be launched by 
Autonomous Systems (ASes) to compromise user anonymity.  There are 
three different types of attacks in this classification.

{\bf Asymmetric Traffic Analysis.} AS-level adversaries can
exploit the asymmetric nature of Internet routing to increase
the chance of observing at least one direction of
user traffic at both ends of the communication.

{\bf Natural Churn.} AS-level adversaries can exploit natural churn in Internet
routing to lie on the BGP paths for more users over
time.

{\bf BGP Hijacks.} Strategic AS-level adversaries can manipulate Internet
routing via BGP hijacks (to discover the users using
specific Tor guard nodes) and interceptions (to perform
traffic analysis).

\section{Measuring Tor's Current State of \\Resiliency to Hijack Attacks}

Because the Tor network is susceptible to network-level adversaries, we aim to quantify how much of the Tor network would be affected by a BGP prefix hijack.  The metrics used help enlighten the community about the state of the Tor network, in terms of how resilient the relays are to hijack attacks.  Additionally, this helps quantify how vulnerable the Tor network is to network-level adversaries in a novel way.  Specifically, we measure:

\begin{itemize}
\item Resiliency of the ASes that contain relays and compare the resiliency of ASes that contain more relays to those that contain few relays.
\item Impact of a BGP prefix hijack on the Tor network.
\item Probability of any given Tor relay being decieved by a BGP prefix hijack.
\end{itemize}

Previous work has tackled these questions using simulations of the entire Internet \cite{lad2007understanding}.  We build off of this work by applying these metrics to the Tor network.  First, we measure the resiliency of Tor-related ASes to equal-length prefix hijack attacks.  To do so, we use an Internet topology~\cite{caida} to get all of the AS relationships, and construct and AS-level graph.  We identify the Tor-related ASes and simulate prefix hijacks on the graph. Our methodology is:

\begin{enumerate}
\item Construct an AS-level graph from an Internet topology.
\item Identify ASes that have at least one Tor relay.
\item Calculate the number of equally preferred paths from AS A to AS B, where AS B is a Tor-related AS.
\item Calculate the number of equally preferred paths from AS A to AS C, where AS C is the attacking (hijacking) AS.
\item Calculate resiliency using the equation in \cite{lad2007understanding}.
\end{enumerate}

We follow this methodology for every AS A $\neq$ a Tor-related AS, every AS C $\neq$ a Tor-related AS, and for every AS B $=$ Tor-related AS.

\cite{lad2007understanding} explains the probability of a node $v$ being deceived by a given false origin $a$ announcing a route that belongs to true origin $t$:

\[\beta(a,t,v) = \frac{p(v,a)}{p(v,a) + p(v,t)}\]

where $p(v,a)$ is the number of equally preferred paths from node $v$ to false origin $a$ and $p(v,t)$ is the number of equally preferred paths from node $v$ to true origin $t$.  Using this probability, the same researchers introduced the resiliency metric -- the resilience of a node $t$ is the fraction of nodes that believe the true origin $t$ given an arbitrary  hijack against $t$:

\[R(t) = \sum_{a \in N} \sum_{v \in N} \frac{\beta(t,a,v)}{(N-1)(N-2)}\]

where N is the total number of ASes.

%There is also space for metrics regarding specific relays' resiliency to BGP prefix hijack attacks, as well as metrics regarding BGP prefix interception attacks (AS- and relay-level).  

%  We also plan to specifically look at the resiliency of guard relays (as a group) as well as exit relays (as a group). 

In addition to measuring the current state of resilience, we measure how the resilience of Tor relays has changed over the years.  We answer the following questions:

\begin{itemize}
\item Have Tor relays become more resilient since the initial network was built?
\item How fast does relay resilience change? 
\end{itemize}

We plan to answer the first question by calculating the given resilience metrics for each past year - similar to a longitudinal study of Tor relay resiliency.  We plan to answer the second question by calculating the given resilience metrics each week for the next couple of months.  The results from answering the first question will also help us answer the second question.

Next we measure the resiliency of Tor-related ASes to prefix interception attacks.  We modify the methodology from above for this measurement in the following way:

\begin{enumerate}
\item Construct an AS-level graph from an Internet topology.
\item Identify ASes that have at least one Tor relay.
\item Calculate the number of equally preferred paths from AS A to AS B, where AS A $\neq$ AS B, AS A and AS B are not Tor-related ASes, and there must be a Tor relay on the path from AS A to AS B.
\item Calculate the number of equally preferred paths from AS A to AS B, where AS A $\neq$ AS B, AS A and AS B are not Tor-related ASes, there must an AS C (intercepting AS) on the path from AS A to AS B, and no Tor-related AS on the path from AS A to AS B.
\item Calculate resiliency using the equation described above.
\end{enumerate}

Similarly, we measure how this resilience to interception attacks has changed over time.

\subsection{Recent Hijacks in the Wild}

There have been a number of prefix hijack attacks in the past year.  We plan to analyze the BGP routing announcements and withdrawals to find the prefixes that were hijacked and compare them to the list of Tor relay IP addresses at the time.  This will give us information about whether or not prefix hijacks (or routing leaks) in the past year have affected Tor relays.

Some of the hijacks/leaks include: 

\begin{itemize}
\item On November 6th, 2015, AS9498 (BHARTI Airtel Ltd.) hijacked about 16,000 prefixes~\cite{indiahijack}.
\item On June 12th, 2015, AS4788 Telekom Malaysia started to announce about 179,000 of prefixes to Level3 (AS3549, the Global crossing AS)~\cite{malaysialeak}.
\item On March 27th, 2015, a BGP traffic optimizer leaked prefixes, which resulted in more than 7,000 new more-specific prefixes affecting roughly 280 Autonomous Systems, including large networks such as Rogers Cable, Telstra, Telenor, KDDI, BT-UK, Orange, Deutsche Telekom , Sprint, China Telecom, SHAW, LGI-UPC, AT\&T, Comcast, Amazon, Internap, Time Warner Cable, Choopa, Syrian Telecommunications and many more~\cite{bgpoptimizer}.
\end{itemize}

\subsection{Resiliency of ASes against BGP Hijacks}

We obtained the list of Tor guard/exit relays from Tor consensus in December 2015 and retrieved their belonging ASes. Then, we downloaded the AS topology published by CAIDA in December 2015. We evaluated the resiliency of each of the ASes which contain Tor guard/exit relays based on the AS topology using the method in ~\cite{lad2007understanding}. 

The AS topology contains 52680 ASes, in which 612 ASes contain a total of 2548 Tor guard/exit relays. We simulated \emph{all} possible hijacking scenarios against each of the 612 ASes, totaling $52679 \times 612 = 32,239,548$ prefix hijacks. As stated in ~\cite{lad2007understanding}, the resiliency of an AS t is calculated by:
\begin{equation}
R(t) = \sum_{a \in N} \sum_{v \in N} \frac {\bar{\beta}(t,a,v)} {(N-1)(N-2)}
\end{equation}

in which $N$ is the set of all ASes, and $\bar{\beta}(t,a,v)$ is defined as:
\begin{equation}
\bar{\beta}(t,a,v) = \frac {p(v,t)} {p(v,t) + p(v,a)}
\end{equation}

in which $p(v,n)$ is the number of equally preferred paths from node $v$ to node $n$ given the routing policy and path lengths. Thus, we first calculate the resiliency of each Tor AS from each of the 52680 ASes as the \emph{source} AS as following, and then sum up the resulting resiliency $R$ to obtain the total resiliency for each of the Tor ASes. 

\begin{algorithmic}
\Function{CalcResilience}{graph $G$, node $t$}
    \State \Call{CalcPathsFromNode}{$G,t$}
    \State $zeros(R)$
    \For{each reachable node $v$ from node $t$} 
    	\If{node $v$ contains Tor guard/exit relays}
		\State $n \gets $ number of less preferred nodes than node $v$
		\State $R[v] \gets n + (\bar{\beta}(v,a,t)$ $\forall$ equally preferred node a)
	\EndIf
    \EndFor
    \State \Return $R$
\EndFunction
\end{algorithmic}

Note that, the $\Call{CalcPathsFromNode}{G,t}$ step above requires AS-level path predictions. Previous works have shown that AS level paths are determined mainly based on two preferences (citation here):\\
\textbf{Local Preference} [more text here]\\
\textbf{Shortest Path} [more text here]\\
Further more, the AS paths should also have the \emph{valley free} property (citation here), meaning that  [summarize here about the property]. Thus, we use depth first search to traverse the graph from a given source node based on this property and the preferences. We first explore provider-customer paths, which are the most preferred; next, we explore one peer-to-peer path followed by a sequence of provider-customer paths, which are the next preferred; finally, we explore customer-provider paths followed by an optional peer-to-peer path and then followed by a sequence of provider-customer paths. Note that, nodes are explored in the most preferred to least preferred order, and those which are explored in the same step are equally preferred. This ordering will help the above resiliency calculation. 

\begin{algorithmic}
\Function{CalcPathsFromNode}{graph $G$, node $t$}
	\State [to be filled]
\EndFunction
\end{algorithmic}

The results show that [I will insert a plot here of the 612 Tor ASes here of resiliency v.s. number of relays].

\section{Proactive Approaches}
We take three different proactive approaches to counter RAPTOR attacks: 1) convincing relay operators to announce the relay in a /24 prefix, 2) analyzing the feasibiity of having a static route to guard relays, and 3) introducing a new path selection algorithm that minimizes the likelihood that a client sees a hijacked route in the case that her guard is hijacked.

\subsection{Using /24 Prefixes}

Sun \emph{et al.} \cite{sun2015raptor} recently found that >90\% of BGP prefixes hosting relays are
shorter than /24, making them vulnerable to a more-specific BGP prefix attack. Thus, one quick way to make Tor relays more resilient to such active routing attacks is to announce /24 prefix covering Tor relays. In order to make real world impact of this approach, we plan to start a campaign by contacting network operators whose prefixes contain Tor relays, and asking them to announce a more specific /24 prefix covering the relay. 

\subsection{Static Routing (or path protection mechanisms?)}

There has been progress in protecting certain BGP paths by using static routes, or other protection mechanisms.  We plan to explore how difficult it is to create static routes to guard relays in order to help previent prefix hijacks that aim to hijack a guard relay.  

In the past, there has been work on protecting routes to top-level DNS servers.  This was done by a fairly aggressive set of filters to be a little more more conservative in accepting alternate routes to the DNS servers~\cite{staticroute}.

\subsection{Path Selection of Guard Relay}

Guard relay is at an important position that it has direct connection with the Tor client. Thus, securing the guard relay would be our first step. It has been shown that AS-level adversaries can launch a more-specific prefix attack to intercept the Tor traffic from the guard relay to the malicious AS \cite{sun2015raptor}, and this can be potentially prevented by advertising /24 prefixes. However, even if the guard relay belongs to a /24 prefix, it is still subject to an equally-specific prefix attack. Unlike more-specific attacks which spread through the whole internet, equally-specific attacks can only affect connections within a small range - i.e., ASes that are within a certain number of hops away, depending on the influence of the announcing AS. Thus, picking a guard relay that is relatively close to the client AS could make it more resilient to such equally-specific attack.

On the other hand, we also don't want to pick a guard relay that is too close - in an extreme case, picking a guard within the same AS as the client will reveal client location to the adversary. 

Therefore, we plan to develop new path selection algorithm that incorporates this aspect, picking a guard relay that satisfies the optimal balance between resilience to routing attacks and privacy protection. 

\section{Reactive Approaches}
\annie{I can add information about the BGP monitoring framework, as well as the small study on accuracy of ASNs in Team Cymru or other registries (to motivate our use of them in the monitoring framework)}

\subsection{A Live Monitoring Framework for Tor}
The live monitoring framework aims at detecting suspicious routing attacks that affect Tor relays, and then react correspondingly to alert Tor clients of the scenario. The monitoring framework consists of two parts: BGP monitoring and Traceroute monitoring.

{\bf Relay Info from Tor Consensus} Tor consensus releases up-to-date information for current running relays every hour. Our system automatically grabs this consensus data once it's updated. We focus on guard relays and exit relays, which reside at the two ends of the communication path and can easily be the target of an adversary. Further more, since we focus on AS-level adversaries, so it is unnecessary to monitor each individual relay by its IP address. But instead, we monitor the /24 prefixes which contain Tor guard and exit relays. Note that, there is no need to monitor a more specific prefix than /24, since generally /24 is the longest prefix accepted in BGP announcement. \\
Thus, we construct a live monitoring database which is being updated every hour with latest Tor relay data. The table contains the following fields:
\begin{center}
\begin{tabular}{ p{8mm} | p{1.4cm} | p{1.3cm} | p{1.3cm} | p{1.3cm}}
  \hline			
  $/24$ Prefix & Total Bandwidth & Number of Guards & Number of Exits & Timestamp \\
  \hline  
\end{tabular}
\label{tab:relayinfo}
\end{center}
Each /24 prefix that contains any guard/exit relays will have one entry in the table, and the list of /24 prefixes will be used for our BGP and Traceroute monitoring frameworks, which we describe in the following. 

{\bf BGP Monitoring Framework} monitors the control plane of internet routing. We plan to collect live BGP announcements data from sources like RouteViews, RIPE-RIS, etc., as well as the most current up-to-date Tor relay information. Then, we will extract the Tor-related BGP activity from the data, and check if any activity looks anomalous.  These anomalies can be detected by developing certain heuristics, such as the amount of time that a BGP path is used or the frequency that a path us announced; if certain anomalies fall under a threshold for a given heuristic, they should be flagged as potential attacks. This analysis will require saving offline relay bgp info for some period of time.  This framework should help:
\begin{itemize}
\item Differentiate between hijack attacks and interception attacks~\cite{ballani2007study}.
\item Differentiate between attacks and ``normal'' behavior, such as multiple origin AS conflicts, backup paths, etc~\cite{zhao2001analysis}.
%-check resilience of relays to prefix hijacking (see AS resilience work)
\end{itemize}

{\bf Traceroute Monitoring Framework} monitors the data plane of internet routing, i.e., how packets travel through the Internet in reality. The traceroute monitoring framework is used as a verification mechanism if the BGP monitoring framework flags certain behavior. There may be many false positives in detecting hijack/interception attacks due to the nature of BGP.  With many false positives, the traceroute monitoring framework will be used often for verification - this raises a question of optimization, which we will also address.

Our Traceroute monitoring framework retrieves updated Tor relay data hourly from the database described above. Since running large number of continuous traceroutes to relay IP addresses may create unnecessary extra traffic to the Tor network, so we selectively monitor a subset of prefixes at certain frequency rates when there is no anomaly from BGP monitoring data, and when there is suspicious activity report by BGP monitoring, the traceroute monitoring can be "triggered" to target the suspicious prefix announcement to verify the anomaly. 
\begin{itemize}
\item Selectively monitor relays of interest.\\
A /24 prefix can be evaluated based on several factors: (1) total combined bandwidth of Tor relays it covers, denoted as $b_i$ for prefix $i$; (2) total number of guard relays it covers, denoted as $g_i$; (3) total number of exit relays it covers, denoted as $e_i$; and (4) resilience of the prefix to BGP hijack/interception attacks. These factors can make the prefix an attractive target to adversaries. Using these factors, we want to formulate the overall security of the system as following, and the goal is to find the monitoring frequency $f_i$ for each relay $i$ that maximizes the overall security of the network. \\
\yixin{not including the resilience factor for now. can add it after we figure out the resilience stuff.}
\begin{align}
\max \sum_{i=1}^N & \frac {\log {(f_i + 1)}} {b_i + g_i + e_i}\\
\text{s.t. } &\sum_{i=1}^N f_i \leq F\\
&0 < f_i \leq (notsure), \forall i
\end{align}
Given the solution, we use a collection of Planetlab nodes located in different ASes to send traceroutes to the prefix at its frequency rate. 
\item Monitoring target prefixes triggered by BGP.\\
If we detect any anomaly from the BGP monitoring framework, we will immediately send traceroutes to the suspicious prefixes to verify whether there is truly a path change happening on the data plane to the Tor relays. 
\item Detecting anomaly from Traceroutes\\
Even when there is no suspicious activity reported by BGP monitoring, it is also possible we detect anomaly from our selective traceroute monitoring. We keep track of the past traceroute monitoring results in a database table, as following:
\begin{center}
\begin{tabular}{ p{8mm} | p{8mm} | p{6mm} | p{1cm} | p{1.1cm} | p{8mm}}
  \hline			
  Source Prefix & Dest Prefix & AS Path & Time Created & Time Last Updated & Current \\
  \hline  
\end{tabular}
\label{tab:pathinfo}
\end{center}
With this table, we will be able to compare the current AS path with past AS paths to detect any path changes, which may indicate a hijack event happening. 

\end{itemize}


\subsection{Framework Evaluation}
The live monitoring framework will be evaluated on a number of characteristics, including false positive rate, false negative rate, as well as performance and overhead.

\subsection{Deployment Experience}

\section{Related Work}

{\bf BGP Attacks and Security.}
BGP attacks are well-studied, particularly prefix hijack and interception attacks~\cite{ballani2007study, mcarthur2009stealthy, zhang2012studying}.  Arnbak, et al. showed that prefix interceptions could be used by nation-states as a way to conduct surveillance on their citizens \cite{arnbak2014loopholes}.  It's also known that routing anomalies can lead to network snapshots that look similar to attack scenarios.  These are due to a range of routing policies, misconfigurations, and multiple origin AS conflicts~\cite{caesar2005bgp, mahajan2002understanding, zhao2001analysis}.  

The research community has contributed a number of protocols to help secure interdomain routing \cite{boldyreva2012provable, chan2006modeling, gill2011let, hu2004spv, zhang2009hc, van2007interdomain}.  Unfortunately, it has also been shown that partial deployment of secure interdomain routing protocols does not provide much security \cite{lychev2013bgp}.

There is also a large body of research with the goal of defending against and detecting prefix hijacks and interceptions.  These include defensive and detection tools~\cite{lad2006phas, hu2007accurate, shi2012detecting, zhang2008ispy, zheng2007light, sriram2009comparative, zhang2007practical}, as well as mechanisms such as PGBGP, which allow network administrators more time to determine if an attack is happening before using new routes~\cite{karlin2006pretty}.  There has also been research not only on detecting attacks, but on determining the location of the attacker~\cite{qiu2009locating}.  Qui, et al. detected any bogus routes, not just hijacks or interceptions~\cite{qiu2007detecting}.  In addition to detection algorithms, there has been research in visualization of real-time detection algorithms \cite{teoh2006bgp}.  Our work does not aim to contribute a new hijack detection tool, but rather compliments existing tools by applying a monitoring framework to the Tor network.  

{\bf BGP Attack Resiliency.}
Prior research on prefix hijack attack resilience has been simulated on the Internet for equal-length prefix hijacks \cite{lad2007understanding}.  They find that customers of Tier-1 ASes are the most resilient and also create the most impact (if they were to hijack a prefix).  There has been some related work in relating hijack attacks to the Internet hierarchy \cite{zhao2012relation, zhao2012analysis}.  This differs from our work; we focus on the resilience of ASes that contain Tor relays, as well as measure the resilience of guard relays and exit relays as groups.

{\bf Network Adversaries on Tor.}
Network-level adversaries are known in anonymity networks. Feamster and Dingledine \cite{feamster2004location} first investigated AS-level path in anonymity networks, which showed that some AS could appear on nearly 30\% of entry-exit pairs. Murdoch and Zielinski \cite{murdoch2007sampled} later demonstrated the threat posed by network-level adversaries who can deanonymize users by performing traffic analysis. Furthermore, Edman and Syverson \cite{edman2009awareness} demonstrated that even given the explosive growth of Tor during the past years, still about 18\% of Tor circuits result in a single AS being able to observe both ends. In 2013, Johnson \emph{et al.} \cite{johnson2013users} evaluated the security of Tor users over a period of time, and the result indicated that a network-level adversary with just low bandwidth cost can deanonymize any users within three months with over 50\% probability and within six months with over 80\% probability.

{\bf AS-level Tor Path Selection.}
The existence of network-level adversaires urges the need to incorporate AS-awareness path selection in Tor. In 2012, Akhoondi \emph{et al.} \cite{akhoondi2012lastor} proposed LASTor, a Tor client which takes into account AS-level path and relay locations in path selection, although LASTor neglected relay capacity and its AS resilience to active attacks. Recently, Nithyanand \emph{et al.} \cite{starov2015measuring} constructed a new Tor client, Astoria, which adopted a new path selection algorithm which considered more aspects - relay capacity, asymmetric routing, colluding ASes, etc.. However, Astoria only considers a passive AS-level attacker, while does not evaluate the AS resilience to an active routing attack.

Towards this goal, it is important to understand AS-level internet topology and network path predictions. Lad \emph{et al.} \cite{lad2007understanding} investigated the relation between internet topology and prefix hijacking, and provided a metric for evaluating AS resilience to active prefix hijack attacks. Although, the study was conducted in 2007 when there were far less ASes than now. Recently, Juen \emph{et al.} \cite{juen2014defending} performed a measurement study using Tracecroutes on network-level paths that Tor traffic actually get routed through. 


\section{Future Work}

%\item Quantify if Tor relays have become more resilient since the initial network was built.
%\item Quantify how fast relay resilience changes.

\section{Conclusion}

In this work, we have presented countermeasures to a set of RAPTOR attacks - attacks that 
involve AS-level adversaries.  In particular, our proposed countermeasures target routing 
manipulations such as BGP prefix hijack and interception attacks.  

First, we evaluated the Tor network for it's current state of resilience to hijack attacks.  
We saw some ASes have a much higher resilience than others, and we compared this to the number 
of relays that each AS contains.  

Then we presented proactive and reactive countermeasures.  The proactive countermeasures included 
having operators announce relays in a /24, using a static route to guard relays, and introducing 
a new path selection algorithm for a client to a guard relay.  The reactive countermeasures included 
a live monitoring framework with both a data plane and control plane component, as well as a set of 
heuristics that help identify suspicious routing changes.  


{\footnotesize \bibliographystyle{acm}
\bibliography{sigproc.bib}}

%\theendnotes

\end{document}







