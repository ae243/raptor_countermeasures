\section{Future Work}

This is an important research area, and there is still much more to do.  There are three main areas where we wish to further this work: quantification of resiliency, monitoring framework, and qualitative suggestions for relay operators.

{\bf Resiliency.}  
We plan to measure how resilience to interception attacks has changed over time (similar to our longitudinal analysis of hijack resiliency).

We also want to give some perspective to the resilience values by running the resilience calculations on a topology from a time when a known attack has occurred.  We can then check what the resilience is of the AS that was actually hijacked - this will give up a notion of how accurate resiliency is.  (This can be done multiple times to increase our confidence in the resiliency metric.)

{\bf Monitoring Framework.}  

As of now, our live monitoring framework consists of two parts: data plane and control plane.  One of the most important next steps is to connect both parts and run the system as one large framework.  The BGP monitoring framework could run consistently, and when any suspicious announcements are flagged (either by the heuristics or the AS comparison), then this could trigger the traceroute monitoring framework.  

Other future work includes developing methods to detect interception attacks - this has been shown to be difficult and accurate detection is still an open problem~\cite{ballani2007study}.  Furthermore, making the already implemented heuristics and techniques more accurate and precise is still to be done.  

Lastly, the monitoring framework needs methods for evaluation.  Metrics such as false positive rate, true positive rate, time and performance, will all be beneficial to showing the importance of the framework, as well as for improving the framework.  

{\bf Qualitative Suggestions.}

Additional work includes setting up our own set of relays at Princeton University and working with the necessary operators to announce the relays in their own /24 network and think of the possibility of static routing to the guard relay.
