\section{Related Work}

{\bf BGP Attacks and Security.}
BGP attacks are well-studied, particularly prefix hijack and interception attacks~\cite{ballani2007study, mcarthur2009stealthy, zhang2012studying}.  Arnbak, et al. showed that prefix interceptions could be used by nation-states as a way to conduct surveillance on their citizens \cite{arnbak2014loopholes}.  It's also known that routing anomalies can lead to network snapshots that look similar to attack scenarios.  These are due to a range of routing policies, misconfigurations, and multiple origin AS conflicts~\cite{caesar2005bgp, mahajan2002understanding, zhao2001analysis}.  

The research community has contributed a number of protocols to help secure interdomain routing \cite{boldyreva2012provable, chan2006modeling, gill2011let, hu2004spv, zhang2009hc, van2007interdomain}.  Unfortunately, it has also been shown that partial deployment of secure interdomain routing protocols does not provide much security \cite{lychev2013bgp}.

There is also a large body of research with the goal of defending against and detecting prefix hijacks and interceptions.  These include defensive and detection tools~\cite{lad2006phas, hu2007accurate, shi2012detecting, zhang2008ispy, zheng2007light, sriram2009comparative, zhang2007practical}, as well as mechanisms such as PGBGP, which allow network administrators more time to determine if an attack is happening before using new routes~\cite{karlin2006pretty}.  There has also been research not only on detecting attacks, but on determining the location of the attacker~\cite{qiu2009locating}.  Qui, et al. detected any bogus routes, not just hijacks or interceptions~\cite{qiu2007detecting}.  In addition to detection algorithms, there has been research in visualization of real-time detection algorithms \cite{teoh2006bgp}.  Our work does not aim to contribute a new hijack detection tool, but rather compliments existing tools by applying a monitoring framework to the Tor network.  

{\bf BGP Attack Resiliency.}
Prior research on prefix hijack attack resilience has been simulated on the Internet for equal-length prefix hijacks \cite{lad2007understanding}.  They find that customers of Tier-1 ASes are the most resilient and also create the most impact (if they were to hijack a prefix).  There has been some related work in relating hijack attacks to the Internet hierarchy \cite{zhao2012relation, zhao2012analysis}.  This differs from our work; we focus on the resilience of ASes that contain Tor relays, as well as measure the resilience of guard relays and exit relays as groups.

{\bf Network Adversaries on Tor.}
Network-level adversaries are known in anonymity networks. Feamster and Dingledine \cite{feamster2004location} first investigated AS-level path in anonymity networks, which showed that some AS could appear on nearly 30\% of entry-exit pairs. Murdoch and Zielinski \cite{murdoch2007sampled} later demonstrated the threat posed by network-level adversaries who can deanonymize users by performing traffic analysis. Furthermore, Edman and Syverson \cite{edman2009awareness} demonstrated that even given the explosive growth of Tor during the past years, still about 18\% of Tor circuits result in a single AS being able to observe both ends. In 2013, Johnson \emph{et al.} \cite{johnson2013users} evaluated the security of Tor users over a period of time, and the result indicated that a network-level adversary with just low bandwidth cost can deanonymize any users within three months with over 50\% probability and within six months with over 80\% probability.

{\bf AS-level Tor Path Selection.}
The existence of network-level adversaires urges the need to incorporate AS-awareness path selection in Tor. In 2012, Akhoondi \emph{et al.} \cite{akhoondi2012lastor} proposed LASTor, a Tor client which takes into account AS-level path and relay locations in path selection, although LASTor neglected relay capacity and its AS resilience to active attacks. Recently, Nithyanand \emph{et al.} \cite{starov2015measuring} constructed a new Tor client, Astoria, which adopted a new path selection algorithm which considered more aspects - relay capacity, asymmetric routing, colluding ASes, etc.. However, Astoria only considers a passive AS-level attacker, while does not evaluate the AS resilience to an active routing attack.

Towards this goal, it is important to understand AS-level internet topology and network path predictions. Lad \emph{et al.} \cite{lad2007understanding} investigated the relation between internet topology and prefix hijacking, and provided a metric for evaluating AS resilience to active prefix hijack attacks. Although, the study was conducted in 2007 when there were far less ASes than now. Recently, Juen \emph{et al.} \cite{juen2014defending} performed a measurement study using Tracecroutes on network-level paths that Tor traffic actually get routed through. 

