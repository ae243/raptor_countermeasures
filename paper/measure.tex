\section{Measuring Tor's Current State of Resiliency to Hijack Attacks}
\annie{I'll work on this section - cleaning up and more specifically stating what we'll do}
While the Tor network is susceptible to network-level adversaries, and we propose a proactive path selection defense, we aim to quantify how much of the Tor network would be affected by a BGP prefix hijack.  Proposed metrics will help enlighten the community about the state of the Tor network, in terms of how resilient the relays are to hijack attacks.  Specifically, we aim to measure:

\begin{itemize}
\item Resiliency of the ASes that contain relays and compare the resiliency of ASes that contain more relays to those that contain few relays.
\item Impact of a BGP prefix hijack on the Tor network.
\item Probability of any given Tor relay being decieved by a BGP prefix hijack.
\end{itemize}

Previous work has tackled these questions using simulations of the entire Internet \cite{lad2007understanding}.  We will build off of this work by applying these metrics to the Tor network.  There is also space for metrics regarding specific relays' resiliency to BGP prefix hijack attacks, as well as metrics regarding BGP prefix interception attacks (AS- and relay-level).  

These metrics will help quantify how vulnerable the Tor network is to network-level adversaries in a novel way.  We plan to look at the resiliency of guard relays (as a group) as well as exit relays (as a group). 

\subsection{Recent Hijacks in the Wild}
\annie{I'll look into any recent (since the RAPTOR paper) attacks that have affected Tor relays}

