% TEMPLATE for Usenix papers, specifically to meet requirements of
%  USENIX '05
% originally a template for producing IEEE-format articles using LaTeX.
%   written by Matthew Ward, CS Department, Worcester Polytechnic Institute.
% adapted by David Beazley for his excellent SWIG paper in Proceedings,
%   Tcl 96
% turned into a smartass generic template by De Clarke, with thanks to
%   both the above pioneers
% use at your own risk.  Complaints to /dev/null.
% make it two column with no page numbering, default is 10 point

% Munged by Fred Douglis <douglis@research.att.com> 10/97 to separate
% the .sty file from the LaTeX source template, so that people can
% more easily include the .sty file into an existing document.  Also
% changed to more closely follow the style guidelines as represented
% by the Word sample file. 

% Note that since 2010, USENIX does not require endnotes. If you want
% foot of page notes, don't include the endnotes package in the 
% usepackage command, below.

% This version uses the latex2e styles, not the very ancient 2.09 stuff.
\newcommand{\comment}[1]{}

\documentclass{acm_proc_article-sp}
\usepackage{graphicx}
\usepackage{usenix,epsfig}
\usepackage{url}
\usepackage{listings}
\usepackage{subfigure}
\usepackage{paralist}
\usepackage{textcomp}
\usepackage{xspace}
\usepackage{ifthen}
\usepackage{amsmath}
\usepackage{amssymb}
\usepackage{color}
\usepackage{algpseudocode}
\usepackage{algorithm}

\begin{document}

%don't want date printed
\date{}

\title{\Large \bf Countermeasures for RAPTOR Attacks}
\author{
 {\rm Yixin Sun}\\
 Princeton University
 \and
 {\rm Anne Edmundson}\\
 Princeton University
} % end author

\maketitle

%\thispagestyle{empty}

\section{Problem Motivation}
Tor is a wildly used system for anonymous communications. However, Tor is known to be vulnerable to traffic correlation attacks in which an adversary can correlate the traffic at both ends of the communication to deanonymize the users. Recently, researchers have shown that AS-level adversaries can exploit the dynamics of BGP routing and launch new attacks on Tor \cite{sun2015raptor}, including both passive attacks exploiting routing asymmetry and natural churn, as well as active attacks with BGP prefix hijacking/interception. Thus, these new attacks urge the need to build countermeasures to defend Tor against such malicious AS-level adversaries. In this work, we will focus on developing countermeasures against active BGP routing attacks. 

\section{Novel Ideas/Perspectives}
Goal: provide (effective) countermeasures against active RAPTOR attacks.

\subsection{Path Selection of Guard Relay}
-path selection against active attacks\\
-how to pick a good/close guard relay\\
-how to handle the tradeoff between resilience and privacy - for instance, picking a guard that's too far away makes the client more vulnerable to equally-specific prefix attacks, while picking a guard that's too close may reveal the location information of the client. 

\subsection{Live Monitoring Framework}
{\bf BGP Monitoring Framework}

{\bf Traceroute Monitoring Framework}

{\bf Challenges}

-monitor BGP announcements, use traceroute to verify (becomes optimization problem)\\
-use relay info to selectively monitor relays\\
-differentiate between hijacks and interceptions\\
-differentiate between attacks and legitimate announcements by multi-origin AS\\

NOTE: other ideas
-check resilience of relays to prefix hijacking (see AS resilience work)

\section{Summary of Proposed Contributions}

{\footnotesize \bibliographystyle{acm}
\bibliography{paper.bib}}

%\theendnotes

\end{document}







