% TEMPLATE for Usenix papers, specifically to meet requirements of
%  USENIX '05
% originally a template for producing IEEE-format articles using LaTeX.
%   written by Matthew Ward, CS Department, Worcester Polytechnic Institute.
% adapted by David Beazley for his excellent SWIG paper in Proceedings,
%   Tcl 96
% turned into a smartass generic template by De Clarke, with thanks to
%   both the above pioneers
% use at your own risk.  Complaints to /dev/null.
% make it two column with no page numbering, default is 10 point

% Munged by Fred Douglis <douglis@research.att.com> 10/97 to separate
% the .sty file from the LaTeX source template, so that people can
% more easily include the .sty file into an existing document.  Also
% changed to more closely follow the style guidelines as represented
% by the Word sample file. 

% Note that since 2010, USENIX does not require endnotes. If you want
% foot of page notes, don't include the endnotes package in the 
% usepackage command, below.

% This version uses the latex2e styles, not the very ancient 2.09 stuff.
\newcommand{\comment}[1]{}

\documentclass{acm_proc_article-sp}
\usepackage{graphicx}
\usepackage{usenix,epsfig}
\usepackage{url}
\usepackage{listings}
\usepackage{subfigure}
\usepackage{paralist}
\usepackage{textcomp}
\usepackage{xspace}
\usepackage{ifthen}
\usepackage{amsmath}
\usepackage{amssymb}
\usepackage{color}
\usepackage{algpseudocode}
\usepackage{algorithm}

\begin{document}

%don't want date printed
\date{}

\title{\Large \bf Countermeasures for RAPTOR Attacks}
\author{
 {\rm Yixin Sun}\\
 Princeton University
 \and
 {\rm Anne Edmundson}\\
 Princeton University
} % end author

\maketitle

%\thispagestyle{empty}

\section{Problem Motivation}
Tor is a wildly used system for anonymous communications. However, Tor is known to be vulnerable to traffic correlation attacks in which an adversary can correlate the traffic at both ends of the communication to deanonymize the users. Recently, researchers have shown that AS-level adversaries can exploit the dynamics of BGP routing and launch new attacks on Tor \cite{sun2015raptor}, including both passive attacks exploiting routing asymmetry and natural churn, as well as active attacks with BGP prefix hijacking/interception. Thus, these new attacks urge the need to build countermeasures to defend Tor against such malicious AS-level adversaries. In this work, we will focus on developing countermeasures against active BGP routing attacks. 

\section{Novel Ideas/Perspectives}
There are two potential ways to counter the attacks: (1) Mitigating traffic interception, and (2) Mitigating correlation attacks. However, mitigating correlation attacks usually involve employing extra encryption schemes and/or packet obfuscation, which either requires lots of engineering efforts, or could result in high latency of Tor. Thus, given the constraints of mitigating correlation attacks, we will build countermeasures that focus on mitigating traffic interception, more specifically, against active BGP routing attacks. Our approach includes two parts, as following. 

\subsection{Path Selection of Guard Relay}
Guard relay is at an important position that it has direct connection with the Tor client. Thus, securing the guard relay would be our first step. It has been shown that AS-level adversaries can launch a more-specific prefix attack to intercept the Tor traffic from the guard relay to the malicious AS \cite{sun2015raptor}, and this can be potentially prevented by advertising /24 prefixes. However, even if the guard relay belongs to a /24 prefix, it is still subject to an equally-specific prefix attack. Unlike more-specific attacks which spread through the whole internet, equally-specific attacks can only affect connections within a small range - i.e., ASes that are within a certain number of hops away, depending on the influence of the announcing AS. Thus, picking a guard relay that is relatively close to the client AS could make it more resilient to such equally-specific attack. \\
On the other hand, we also don't want to pick a guard relay that is too close - in an extreme case, picking a guard within the same AS as the client will reveal client location to the adversary. \\
Therefore, we plan to develop new path selection algorithm that incorporates this aspect, picking a guard relay that satisfies the optimal balance between resilience to routing attacks and privacy protection. 

\subsection{Live Monitoring Framework}
The live monitoring framework aims at detecting suspicious routing attacks that affect Tor relays, and then react correspondingly to alert Tor clients of the scenario. The monitoring framework consists of two parts: BGP monitoring and Traceroute monitoring.\\
{\bf BGP Monitoring Framework}
monitors the control plane of internet routing. We plan to collect live BGP announcements data from sources like RouteViews, RIPE-RIS, etc., as well as the most current up-to-date Tor relay information. Then, we will extract the Tor-related BGP activity from the data, and check if any activity hits certain detection threshold. (save offline relay bgp info for a period of time?)\\ 
{\bf Traceroute Monitoring Framework}
monitors the data plane of internet routing, i.e., how packets travel through in real. \\
-monitor BGP announcements, use traceroute to verify (becomes optimization problem)\\
-use relay info to selectively monitor relays\\
{\bf Challenges}\\
-differentiate between hijacks and interceptions\\
-differentiate between attacks and legitimate announcements by multi-origin AS\\
NOTE: other ideas
-check resilience of relays to prefix hijacking (see AS resilience work)

\section{Summary of Proposed Contributions}

{\footnotesize \bibliographystyle{acm}
\bibliography{paper.bib}}

%\theendnotes

\end{document}







