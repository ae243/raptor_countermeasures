\section{Background}
Here we discuss some background on the Tor network as well as RAPTOR attacks~\cite{sun2015raptor}.

\subsection{Tor}
To communicate with a destination, Tor clients establish
layered circuits through three subsequent Tor relays. The
three relays are referred to as: entry (or guard) for the
first one, middle for the second one, and exit relay for
the last one. To load balance its traffic, Tor clients select
relays with a probability that is proportional to their
network capacity. Encryption is used to ensure that each
relay learns the identity of only the previous hop and the
next hop in the communications, and no single relay can
link the client to the destination server.

It is well known that if an attacker can observe the
traffic from the destination server to the exit relay as well
as from the entry relay to the client (or traffic from the
client to the entry relay and from the exit relay to the
destination server), then it can leverage correlation between
packet timing and sizes to infer the network identities
of clients and servers (end-to-end timing analysis).
This timing analysis works even if the communication is
encrypted.

\subsection{RAPTOR Attacks}
RAPTOR attacks are a suite of attacks that can be launched by 
Autonomous Systems (ASes) to compromise user anonymity.  There are 
three different types of attacks in this classification.

{\bf Asymmetric Traffic Analysis.} AS-level adversaries can
exploit the asymmetric nature of Internet routing to increase
the chance of observing at least one direction of
user traffic at both ends of the communication.

{\bf Natural Churn.} AS-level adversaries can exploit natural churn in Internet
routing to lie on the BGP paths for more users over
time.

{\bf BGP Hijacks.} Strategic AS-level adversaries can manipulate Internet
routing via BGP hijacks (to discover the users using
specific Tor guard nodes) and interceptions (to perform
traffic analysis).
